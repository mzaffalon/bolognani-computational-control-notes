\chapter{System Identification}
\label{sec:system-identification}

LQR and MPC require a state space model of the system and the measurement function
\begin{equation*}
  x_{t+1} = f(x_t,u_t),\quad y_t = h(x_t,u_t).
\end{equation*}
Obtaining a model requires determining
\begin{itemize}
\item the Markovian state $x$, so that $x_{t+1}$ is determined by $x_t$ and $u_t$;
\item the dynamics $f(x,u)$ of the system;
\item the measurement function $h(x,u)$.
\end{itemize}
In the rest of this chapter we only consider the case of a noiseless linear time-invariant update and observation:
\begin{equation}
  \label{eq:sysid-dynamics}
  \begin{aligned}
    x_{t+1} &= Ax_t + Bu_t \\
    y_t &= Cx_t + Du_t.
  \end{aligned}
\end{equation}
% where an initial state $x_0$ evolves according the relationship
% \begin{equation*}
%   x_t = \underbrace{A^tx_0}_{\text{initial response}} + \underbrace{\sum_{j=0}^{t-1} A^{t-1-j}Bu_j}_{\text{convolution}}
% \end{equation*}
% and the detected output is
% \begin{equation*}
%   y_t = CA^tx_0 + \sum_{j=0}^{t-1} CA^{t-1-j}Bu_j + \underbrace{Du_t}_{\text{feed through}}.
% \end{equation*}

State space representations are not unique: under the invertible transformation $T$ such that $x=Tw$, the system eq.~\eqref{eq:sysid-dynamics} becomes\footnote{I introduce $D$ which is not in the lecture but is in the tutorials.}
\begin{equation*}
  \begin{aligned}
    w_{t+1} &= T^{-1}ATw_t + T^{-1}Bu_t \\
    y_t &= CTw_t + Du_t
  \end{aligned}.
\end{equation*}
Nevertheless the input-output behavior, the relation between the input $u$ and the output $y$, remains the same. This can be seen by checking that their response is the same for the same input, \textit{e.g.} an input response applied to the $k$ input
\begin{equation*}
  u_{k,0} = e_k\delta_t,\quad \delta_t =
  \begin{cases}
    1 & \text{ for } t=0 \\
    0 & \text{ otherwise}
  \end{cases}.
\end{equation*}
Assuming a system at rest, we have for the state $x$ the dynamics
\begin{equation*}
  x_0=0,\ x_1=Be_k,\ x_2=ABe_k,\ x_3=A^2Be_k,\ \ldots
\end{equation*}
whereas for $w$
\begin{equation*}
  w_0=0,\ w_1=T^{-1}Be_k,\ w_2=T^{-1}ABe_k,\ w_3=T^{-1}A^2Be_k,\ \ldots
\end{equation*}
and for both systems, the output $y_t$ is the same,
\begin{equation*}
  y_0=De_k,\ y_1=CBe_k,\ y_2=CABe_k,\ y_3=CA^2Be_k,\ \ldots
\end{equation*}

For a general system with $m$ inputs and $p$ outputs, the sequence of matrices in $R^{p\times m}$ describing the output of the system dynamics
\begin{equation}
  \label{eq:markov-parameters}
  \{D,\ CB,\ CAB,\ CA^2B,\ \ldots\}
\end{equation}
is called the \emph{Markov parameters}. The matrix element $\left[CA^tB\right]_{ij}$ represent the output $i$ to an unit impulse on the input $j$ after a time $t>0$.

\subsection{Controllability and Observability}
\label{sec:controllability-observability}

Consider the problem of driving a system from $x_0=0$ to a desired $\bar{x}\in \mathbb{R}^n$. In one step, input $u_0$ will take the system to
\begin{equation*}
  x_1 = Bu_0
\end{equation*}
that is, to any state in the image of $B$. In two steps, the control sequence $\{u_0,u_1\}$ will drive the system to
\begin{equation*}
  x_2 = Bu_1 + ABu_0
\end{equation*}
and therefore to any state in the image of $\begin{bmatrix}B & AB\end{bmatrix}$. In $n$ steps, the system can be driven to any state in the image of the \emph{controllability matrix}
\begin{equation}
  \label{eq:controllability-matrix}
  \mathcal{C} :=
  \begin{bmatrix}
    B & AB & A^2B & \ldots & A^{n-1}B
  \end{bmatrix}.
\end{equation}
$\mathcal{C}$'s column space is also the largest space that can be reached by the system: the matrix $A^n$ for an additional step is linearly dependent on $I,A,\ldots, A^{n-1}$ by the Cayley-Hamilton theorem. In general $\mathcal{C}$'s column space does not span the whole space $\mathbb{R}^n$: this is the case only when $\mathcal{C}$'s rank is $n$.

For a full rank controllability matrix, the desired state $\bar{x}$ can be reached in at most $n$ steps since
\begin{equation*}
  \bar{x} = \mathcal{C}
  \begin{bmatrix}
    u_{t-1} \\ u_{t-2} \\ \vdots \\ u_0
  \end{bmatrix} = \mathcal{C} \bsu
\end{equation*}
by using the control input $\bsu = \mathcal{C}^{-1}\bar{x}$ when the input is scalar or by using the pseudoinverse $\bsu = \mathcal{C}^\dagger\bar{x}$ in the case of a MIMO system, where $\mathcal{C}$ is a fat matrix.

Similarly, consider the problem of detecting the effect of an initial condition $x_0$ on the output of a system with no control input. At $t=0$
\begin{equation*}
  y_0 = Cx_0
\end{equation*}
therefore any $x_0$ in $C$'s kernel is non-observable. At $t=1$, the output is
\begin{equation*}
  y_1 = CAx_0
\end{equation*}
and any $x_0$ in the kernel of $\begin{bmatrix} C \\ CA \end{bmatrix}$ is non-observable. In $n$ steps, any initial state in the kernel of the \emph{observability matrix}
\begin{equation}
  \label{eq:observability-matrix}
  \mathcal{O} :=
  \begin{bmatrix}
    C \\ CA \\ CA^2 \\ \vdots \\ CA^{n-1}
  \end{bmatrix}
\end{equation}
is non-observable. As we have seen for the controllability matrix, an additional step, $CA^n$ does not increase $\mathcal{C}$'s row rank because by the Cayley-Hamilton theorem, $A^n$ is a linear combination of $\{I, A,\ldots, A^{n-1}\}$.

We can use the observability matrix to find the initial state $x_0$ given the output sequence $\bsy = \begin{bmatrix}
    y_0 & y_1 & \ldots & y_{n-1}
  \end{bmatrix}^\top$, since
\begin{equation*}
  \bsy = \mathcal{O}x_0.
\end{equation*}
Assuming $\mathcal{O}$ has rank full, if the system has only one output (\textit{i.e.} the matrix $C$ has only row), $\mathcal{O}$ is square and invertible and the initial state is found by $x_0 = \mathcal{O}^{-1}\bsy$; otherwise, in case of a multiple output system, $\mathcal{C}$ is thin and we need to use $\mathcal{O}$'s pseudoinverse, $x_0=\mathcal{O}^\dagger\bsy$.

% Example: consider the system
% \begin{equation*}
%   y_t = a_1u_{t-1} + a_2u_{t-2} + a_3u_{t-3}
% \end{equation*}
% The minimal state representation is to use the state $x=\begin{bmatrix} u_{t-1} & u_{t-2} & u_{t-3} \end{bmatrix}^\top$ such that
% \begin{equation*}
%   x_{t+1} =
%   \begin{bmatrix}
%     0 & 0 & 0 \\
%     1 & 0 & 0 \\
%     0 & 1 & 0
%   \end{bmatrix}x_t +
%   \begin{bmatrix}
%     1 \\ 0 \\ 0
%   \end{bmatrix} u_t,\quad y_t =
%   \begin{bmatrix}
%     a_1 & a_2 & a_3
%   \end{bmatrix}x_t
% \end{equation*}
% with the controllability matrix
% \begin{equation*}
%   \mathcal{C} =
%   \begin{bmatrix}
%     1 & 0 & 0 \\
%     0 & 1 &
%   \end{bmatrix}
% \end{equation*}


\subsection{System Identification from Data}
\label{sec:ho-kalman-algorithm}

We want to derive a state-space representation exclusively from input-output data. We restrict our attention to systems that are both controllable and observable, so that the relation between states and input/output signals is well-posed (?). This also excludes the case of states that have no effect on I/O (is this not the same as the statement before?) We also want the minimal realization, the minimum number of states relevant for the system.

The Ho-Kalman algorithm allows one to derive a minimal realization of the system in the form of the matrices $A$, $B$, $C$ from impulse response data. The algorithm consists of the following steps:
\begin{enumerate}
\item construct the Hankel matrix $\mathcal{H}$ from the Markov parameters;
\item factorize $\mathcal{H}$ into the matrices $\mathcal{O}$ and $\mathcal{C}$;
\item derive the realization of $A$, $B$, $C$.
\end{enumerate}

\subsubsection{Step 1: Construct $\mathcal{H}$}
\label{sec:construct-Hankel}

The impulse responses of the system are collected as Markov parameters eq.~\eqref{eq:markov-parameters}. The Hankel matrix is a banded matrix
\begin{equation}
  \label{eq:hankel-matrix-definition}
  \mathcal{H} := \mathcal{O}\mathcal{C} =
  \begin{bmatrix}
    CB & CAB & CA^2 B & \ldots & CA^{n-1}B \\
    CAB & CA^2B & \ldots & & CA^nB \\
    CA^2B & \ddots & & & \ldots \\
    \vdots & & & & \vdots \\
    CA^{n-1}B & \ldots & & & CA^{2n-2}B
  \end{bmatrix}
\end{equation}
with the Markov parameters along the diagonals. Only for SISO systems is each entry a scalar, for MIMO systems each entry has size $p\times m$.

We are seeking a minimal representation of a controllable and observable system. Since both $\mathcal{O}$ and $\mathcal{C}$ are full rank, so must $\mathcal{H}$. The rank $n$ is however unknown: the approach is to construct an extended Hankel matrix $\mathcal{H}_{k,\ell}\in \mathbb{R}^{k\times \ell}$ and choose $k$ and $\ell$ larger than the expected value $n$. In doing so, $\mathcal{H}_{k,\ell}$ will be singular. To determine $n$ we select the largest upper left square minor of $\mathcal{H}_{k,\ell}$ that is non-singular.

This is better explained by an example: consider a time-discrete frictionless cart
\begin{equation*}
  \begin{aligned}
    v_{t+1} &= v_t + u_t \\
    z_{t+1} &= z_t + v_t \\
    y_t &= z_t
  \end{aligned} \longrightarrow
  x_t =
  \begin{bmatrix}
    v_t \\ z_t
  \end{bmatrix}\quad A =
    \begin{bmatrix}
      1 & 0 \\ 1 & 1
    \end{bmatrix}\quad B =
    \begin{bmatrix}
      1 \\ 0
    \end{bmatrix}\quad C =
    \begin{bmatrix}
      0 & 1
    \end{bmatrix}
\end{equation*}
subjected to an impulse response $u_t=\delta_t$. The time evolution is
\begin{equation*}
  \begin{matrix}
    v_0 = 0, & z_0 = y_0 = 0 \\
    v_1 = 1, & z_1 = y_1 = 0 \\
    v_2 = 1, & z_2 = y_2 = 1 \\
    v_3 = 1, & z_3 = y_3 = 2 \\
    \ldots & \ldots
  \end{matrix}
\end{equation*}
Let us arbitrarily pick $k=4$ and $\ell=6$. The corresponding extended Hankel matrix is the matrix whose rows are the system output, each row shifted to the left:
\begin{equation*}
  \mathcal{H}_{4,6} =
  \begin{bmatrix}
    y_0 & y_1 & y_2 & y_3 & y_4 & y_5 \\
    y_1 & y_2 & y_3 & \ldots & & \\
    y_2 & y_3 & \ldots & & & \\
    y_3 & \ldots & & & & y_8
  \end{bmatrix} =
  \begin{bmatrix}
    0 & 1 & 2 & 3 & 4 & 5 \\
    1 & 2 & 3 & 4 & 5 & 6 \\
    2 & 3 & 4 & 5 & 6 & 7 \\
    3 & 4 & 5 & 6 & 7 & 8
  \end{bmatrix}.
\end{equation*}
The $1\times 1$ top left submatrix
\begin{equation*}
  \begin{bmatrix} 0\end{bmatrix}
\end{equation*}
has rank zero, the $2\times 2$ top left submatrix
\begin{equation*}
  \begin{bmatrix}
    0 & 1 \\
    1 & 2
  \end{bmatrix}
\end{equation*}
has rank 2 and so have the other square top left submatrices: the rank does not increase if more rows or columns are added. A maximum rank of 2 is to be expected because the system is 2-dimensional.

The computation of the rank is however not sufficiently robust against perturbation of the coefficients: the next section presents a better approach.


% The recipe is to build an extended Hankel matrix of size $k\times \ell$, where $k,\ell $ are larger than $n$
% \begin{equation*}
%   \mathcal{H}_{k,\ell} =
%   \begin{bmatrix}
%     G_1 & G_2 & \ldots & G_\ell \\
%     G_2 & G_3 & \ldots & G_{\ell+1} \\
%     G_3 & \ddots & & \vdots \\
%     \vdots \\
%     G_k & G_{k+1} & \ldots & G_{k+\ell+1}
%   \end{bmatrix}
% \end{equation*}
% and cut it if necessary.

\subsubsection{Step 2: Factorize $\mathcal{H}$}
\label{sec:factorize-Hankel}

Factorize $\mathcal{H}_{k,\ell}$ as the product of two matrices $\mathcal{O}_k$ and $\mathcal{C}_\ell$
\begin{equation}
  \label{eq:extended-observability-controllability-matrices}
  \mathcal{O}_k =
  \begin{bmatrix}
    C \\ CA \\ \vdots \\ CA^{k-1}
  \end{bmatrix},\qquad \mathcal{C}_\ell =
  \begin{bmatrix}
    B & AB & \ldots & A^{\ell-1}B
  \end{bmatrix}.
\end{equation}
The factorization is not unique and different factorizations give different space state representations of the system, all sharing the same input-output response. Indeed consider a change of basis $\tilde{A} = T^{-1}AT$, $\tilde{B} = T^{-1}B$ and $\tilde{C} = CT$. The observability matrix changes as
\begin{equation*}
  \tilde{\mathcal{O}} =
  \begin{bmatrix}
    \tilde{C} \\ \tilde{C}\tilde{A} \\ \vdots
  \end{bmatrix} =
  \begin{bmatrix}
    CT \\ CTT^{-1}AT \\ \vdots
  \end{bmatrix} = \mathcal{O}T
\end{equation*}
whereas the controllability matrix changes as $\tilde{\mathcal{C}} = T^{-1}\mathcal{C}$ but the product $\tilde{\mathcal{O}}\tilde{\mathcal{C}} = \mathcal{O}\mathcal{C}$ remains the same.

SVD is a computationally fast and robust method to obtain the decomposition of the matrix $\mathcal{H}_{k,\ell} = U \Sigma V^\top$
\begin{align*}
  \mathcal{H}_{k,\ell} &=
                \begin{bmatrix}
                  U_1 & U_2
                \end{bmatrix}
                \begin{bmatrix}
                  \Sigma_1 & 0_{n \times (\ell-n)} \\
                  0_{(k-n)\times n} & 0_{(k-n)\times (\ell-n)}
                \end{bmatrix}
                \begin{bmatrix}
                  V_1 & V_2
                \end{bmatrix}^\top = U_1 \Sigma_1 V_1^\top.
\end{align*}
where the dimension $n$ of the diagonal matrix $\Sigma_1$ is the rank of $\mathcal{H}_{k,\ell}$ and the size of the state space. The remaining elements of $\Sigma$ are zero, because increasing $k$ and $\ell$ does not increase $\mathcal{H}_{k,\ell}$ rank.

A balanced factorization is
\begin{equation*}
  \mathcal{H}_{k,\ell} = \underbrace{U_1\sqrt{\Sigma_1}}_{\mathcal{O}_k} \underbrace{\sqrt{\Sigma_1}V_1^\top}_{\mathcal{C}_\ell}.
\end{equation*}

For the frictionless cart
\begin{equation*}
  \Sigma =
  %\begin{bmatrix}
  %  \sqrt{242 + 4\sqrt{3529}} \\ \sqrt{242- 4\sqrt{3529}} \\ 0 \\ 0
  %\end{bmatrix} =
  \begin{bmatrix}
    21.90 \\ 2.092 \\ 0 \\ 0 \end{bmatrix}
\end{equation*}
and the decomposition gives
\begin{align*}
  \mathcal{O}_k &=
  \begin{bmatrix}
    -1.547 &  -1.112 \\
    -2.033 & -0.4826 \\
    -2.519 &  0.1467 \\
    -3.005 &  0.7759 \\
  \end{bmatrix} \\
  \mathcal{C}_\ell &=
  \begin{bmatrix}
    -0.7345 & -1.150 & -1.566 & -1.982 & -2.397 & -2.813 \\
    1.022 & 0.7010 & 0.3800 & 0.0589 & -0.2621 & -0.583
  \end{bmatrix}.
\end{align*}

\subsubsection{Step 3: Obtain the Realization $A$, $B$, $C$}
\label{sec:obtain-system-realization}

$C$ and $B$ are the first entries in $\mathcal{O}_k$ and $\mathcal{C}_\ell$, eq.~\eqref{eq:extended-observability-controllability-matrices}. $A$ can be obtained from the shifted Hankel matrix, obtained discarding the first column of $\mathcal{H}_{k,\ell}$ and adding the subsequent Markov parameters as the last column (or alternatively, discarding the first row and adding the subsequent Markov parameters as the last row)
\begin{align*}
  \mathcal{H}_{k,\ell}^\uparrow :=
  \begin{bmatrix}
    CAB & CA^2B & CA^3 B & \ldots & CA^\ell B \\
    CA^2B & CA^2B & \ldots & & CA^{\ell+1}B \\
    CA^3B & \ddots & & & \ldots \\
    \vdots \\
    CA^kB & \ldots & & & \ldots
  \end{bmatrix} = \mathcal{O}_k A \mathcal{C}_\ell.
\end{align*}
Then $A$ becomes
\begin{equation*}
  A = \mathcal{O}_k^\dagger\mathcal{H}_{k,\ell}^\uparrow \mathcal{C}_\ell^\dagger.
\end{equation*}

For the frictionless cart, we obtain
\begin{equation*}
  A =
  \begin{bmatrix}
    1.202 & -0.2616 \\
    0.156 &  0.7980
  \end{bmatrix},\quad B =
  \begin{bmatrix}
    -0.7345 \\
    1.0220
  \end{bmatrix},\quad C =
  \begin{bmatrix}
    -1.5470 & -1.1118
  \end{bmatrix}
\end{equation*}
The Ho-Kalman algorithm gives a different representation of the matrices $A$, $B$ and $C$\footnote{There is a coordinate transformation $T$ which I cannot find: $A$ is defective: should I be using the Jordan form?} but the same observed dynamics. (How can I see this?)

Note also that the frictionless cart is only marginally stable as $A$'s eigenvalues are both 1. The approach can be also for unstable systems provided the experiment is stopped before the system becomes irrecoverable. Otherwise, system identification on unstable systems must be performed with feedback stabilization, which is outside this introduction.

%%% Local Variables:
%%% mode: latex
%%% TeX-master: "notes"
%%% End:
