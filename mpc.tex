\chapter{Model Predictive Control}
\label{sec:MPC}

Model Predictive Control (MPC), also referred to as receding horizon control, is a process control technique that optimizes a cost function by using a process model for the state evolution with the additional requirements that the control input $\bsu = \{u_0,u_1,\ldots,u_{K-1}\}$ and the states $\bsx = \{x_0,x_1,\ldots,x_{K-1},x_K\}$ also satisfy a set of contraints\footnote{Input constraints often represent physical limits: if the controller does not respect the input constraints, the plant will enforce them. This is in contrast with state constraints that are normally desirable. For this reason, MPC are often setup as optimization problems using hard constraints for the input and some sort of soft constraints for the states~\cite{MPC-diehl}.}:
\begin{equation}
  \label{eq:MPC-general-formulation}
  \begin{aligned}
    \min_{\bsu} &\sum_{k=0}^{K-1} g_k(x_k,u_k) + g_K(x_K),\\
    \text{subject to } & x_{k+1} = f(x_k,u_k) \\
                     & x_0 = X_0 \\
                     & x_k \in \mathcal{X}_k \\
                     & u_k \in \mathcal{U}_k
  \end{aligned}
\end{equation}
where $X_0$ is the initial state and $g_k$ is a non-negative cost function such as a quadratic one
\begin{equation}
  \label{eq:quadratic-stage-cost}
  g_k(x,u) = ||x||_{Q_k} + ||u||_{R_k} %= x^\top Q_k x + u^\top R_k u.
\end{equation}
In the presence of constraints, one cannot find the optimal solution by setting the gradients $\nabla_{u_k} V(\bsx,\bsu)$ of the cost-to-go
\begin{equation}
  \label{eq:cost-to-go}
  V(\bsx,\bsu) = \sum_{k=0}^{K-1} g_k(x_k,u_k) + g_K(x_K)
\end{equation}
to zero as it was the case for the LQR problem. The optimal control input sequence $\bsu^\star = \{u_0^\star,u_1^\star,\ldots,u_{K-1}^\star\}$ and the optimal trajectory $\bsx^\star = \{x_0^\star,x_1^\star,\ldots,x_K^\star\}$ can however be found if the optimization problem is feasible. In open-loop control, the whole optimal sequence of input control inputs $\bsu^\star$ is applied and the system evolves to reach the terminal state $x_K$ after $K$ steps.

In case of model mismatch or state perturbation however, the system will deviate increasingly more from the predicted trajectory $\bsx^\star$. If a measurement of the system state is available, it makes sense to find a new optimal control input vector using the measured state as the initial point for the optimization. (Question: if this is the case, then why not optimizing only the tail such that also in closed loop the trajectory is completed after $K$ steps? )

Closed-loop control therefore applies the following strategy: at time step $i$ with the system in state $x_i$, compute the optimal control input $\bsu_i^\star=\{u_{i,0}^\star,u_{i,1}^\star\ldots,u_{i,K-1}^\star\}$, apply the first element of it, $u_{i,0}^\star$, and discard the remaining elements. At the following time step $i+1$ and with the system having evolved into state $x_{i+1}$, solve the optimization from the initial state $x_{i+1}$ to find the optimal control input $\bsu_{i+1}^\star$, apply the first element $u_{i+1,0}^\star$ and discard the remaining elements. This process is repeated.

The closed-loop approach permits us to solve the optimization problem with an horizon of length $K$ in a receding way, by restarting the control process from the current state. This is possible since the system is Markovian and the current state is all one needs to compute the optimal control for the future.

If the model closely represents the system and in absence of perturbations, the closed-loop approach can be restarted using the predicted state $x_{k+1} = f(x_k,u_k$); in this case the optimal state starting for $i+1$ concides with that evolved from $x_{i,0}^\star$ and under the optimal control $u_{i,0}^\star$: $x_{i+1,0}^\star = f(x_{i,0}^\star,u_{i,0}^\star$).

Either way, closed-loop MPC solves repeatedly the exact same problem eq.~\eqref{eq:MPC-general-formulation}, parametrised by the initial state $x_0=X_0$. Since the problem is the same, the optimal solution $\bsu^\star$ and in particular its first element $u_0^\star$ are independent from the time index $i$ and the index will be dropped in the following.

In general there is no guarantee that the map $x \rightarrow u_0^\star(x)$ is a continuous function even under the assumption of the continuity of $f$ and $g_k$; the cost-to-go $J(\bsx,\bsu)$ is instead continuous in the parameters $\bsx$ and $\bsu$. Moreover, under compactness assumptions on the constraints, the parametric optimization problem has a solution when it is feasible (I do not understand this sentence).

From a computational point of view, the problem eq.~\eqref{eq:MPC-general-formulation} can be efficiently solved if it is convex, which happens for instance if the stage cost $g_k$ is a convex function, the spaces $\mathcal{X}_k$ and $\mathcal{U}_k$ are convex and the dynamics is linear $f(x_k,u_k)=Ax_k+Bu_k$.

%In the following we will assume that the problem is convex and the dynamics is linear.

To summarize: MPC is a static\footnote{A static controller has no memory of the past and depends only on the current statte $x_k$:
  \begin{equation*}
    u_k = \varphi_k(x_k)
  \end{equation*}
  whereas a dynamic controller has memory, such as in the case of a
  PI controller
  \begin{equation*}
    \xi_{k+1} = \xi_k + \varphi_k(x_k),\quad u_k = \psi_k(x_k) + \xi_k.
  \end{equation*}}, nonlinear, time-invariant feedback\footnote{Because $f$ does not depend on $k$.} control-law.


\subsection{Does MPC Realize the Optimal Control Problem Solution?}
\label{sec:MPC-not-an-optimal-control-solver}

Despite MPC solving at each time step $i$ the optimal control problem eq.~\eqref{eq:MPC-general-formulation}, the MPC trajectory does not realize the optimal trajectory $\bsu_i^\star$, because MPC only applies the first control input $u_0^\star$ and discards the rest. Consider the pathological problem
\begin{align*}
  \min_{\bsu} & \sum_{k=0}^{K-1}\frac {1}{k}||u_k||^2 + ||x(K)||^2\\
  \text{subject to } &x_{k+1} = x_k+u_k.
\end{align*}
The optimization finds a solution where the first control input $u_0^\star$ must be zero or the cost explodes and delays the work to minimize the cost to go to a later time. By applying only the first control input, MPC will remain stuck in its initial state and the optimal control solution is never realized. This is so also in absence of disturbance and model mismatch.

From this example, it is clear that MPC is a way to compute a good controller but in no way a robust optimal controller\footnote{Needs explanation: what is a robust optimal controller?}.

\subsection{MPC and Finite-Time LQR}
\label{sec:MPC-finite-time-LQR}

Given the finite-time LQR problem eq.~\eqref{eq:LQR-formulation}, we have seen that the LQR solution computes the matrices $\{P_K, P_{K-1},\ldots P_1\}$ by backward induction and constructs from them the linear feedback law $u=-\Gamma_kx$:
\begin{equation*}
  u_0^\textrm{LQR} = -\Gamma_0x_0, \quad u_1^\textrm{LQR} = -\Gamma_1\left(Ax_0+Bu_0^\textrm{LQR}\right), \quad \ldots
\end{equation*}
Compare this with the MPC approach\footnote{Why are we comparing a finite horizon finite simulation time LQR with MPC that has an infinite \emph{simulation} time? I did not get the point, aside the obvious one that infinite horizon LQR and MPC give the same result. Either clarify or remove this slide.}: the optimal control solution is the same as that of the LQR, since MPC solves the same problem, and finds the same sequence $\{P_K, P_{K-1},\ldots P_1\}$. MPC however applies the linear \emph{time-invariant} feedback law $x=-\Gamma_0x$:
\begin{equation*}
  u_0^\textrm{MPC} = -\Gamma_0x_0, \quad u_1^\textrm{MPC} = -\Gamma_0\left(Ax_0+Bu_0^\textrm{MPC}\right), \quad \ldots
\end{equation*}
In the case of infinite horizon problems, the feedback law is the same, since the feedback law becomes time invariant also for the LQR problem.

%%%%%%%%%%%%%%%%%%%%%%%%%%%%%%%%%%%%%%%%%%%%%%%%%%%%%%%%%%%%
\section{Stability}
\label{sec:mpc-stability}

MPC is a principled way to design static, time-invariant, non-linear feedback control law. While local stability can be analyzed by linearizing the control law around the equilibrium and using the techniques from Control Systems (Nyquist, Bode,\ldots), the tool to investigate global stability of non-linear systems is the Lyapunov theorem.

\subsection{Lyapunov Stability}
\label{sec:lyapunov-stability}

\begin{definition}
  The point $x=0$ is \emph{stable} for the dynamics $x_{k+1} = f(x_k)$ if a small perturbation of the state perturbs the subsequent state trajectory in a continuous manner.

  In mathematical terms,
  \begin{equation}
    \label{eq:lyapunov-stability-stable-equilibrium}
    \forall \epsilon > 0,\ \exists \delta > 0 \text{ such that } ||x_0|| < \delta \rightarrow ||x_k|| < \epsilon\ \forall k \ge 0.
  \end{equation}
  In other words, given a ball of radius $\epsilon$, any trajectory starting from within a ball of radius $\delta$ will never leave the larger ball with radius $\epsilon$.

  An equilibrium is called \emph{asymptotically stable} if it is stable and $\lim_{k\rightarrow \infty} ||x_k||=0$.
\end{definition}

\begin{theorem}[Lyapunov theorem]
  \label{th:lyapunov}
  Given a discrete system with dynamics $x_{k+1} = f(x_k)$ and equilibrium $x=0$ (?), if $W(x)$ is a real-valued function such that
  \begin{equation*}
    \begin{aligned}
      W(0)=0 \textrm{ and } W(x)>0\quad \forall x\neq 0 \\
      W(f(x)) < W(x)\quad \forall x\neq 0
    \end{aligned}
  \end{equation*}
  then the point $x=0$ is asymptotically stable.
\end{theorem}
A survey of formulation of Lyapunov theory for discrete systems is available at \url{https://arxiv.org/abs/1809.05289}.

\subsection{Stability of MPC}
\label{sec:stability-MPC}

To prove asymptotically stability of the closed loop problem\footnote{Note that the open loop is trivially asymptotically stable because it consists of a sequence of finite length $K$.}, we need to find a Lyapunov function $W(x)$ that satisfies theorem~\ref{th:lyapunov}, for the equilibrium point $(x_S,u_S)=(0,0)$.

We consider first the simpler case of stability for the infinite horizon unconstrained problem. The presence of constraints does not alter the validity of the proof. The function
\begin{equation*}
  W(x) = \min_{\bsu} \sum_{k=0}^\infty g_k(x_k,u_k).
\end{equation*}
is a candidate for the Lyapunov function provided that 1. $W(x)$ is only zero at the target equilibrium (such as when $g_k$ is a quadratic objective function) and 2. that it decreases along trajectories of the system. This is the case because the MPC controller finds $(\bsx^\star, \bsu^\star)$ that minimizes the problem
\begin{equation*}
  V^\infty_i(\bsx,\bsu) \equiv \sum_{k=i}^\infty g_k(x_k,u_k).
\end{equation*}
By Bellman's principle, the optimal trajectory minimizes also the ``tail'' cost starting at the subsequent time $i+1$
\begin{equation*}
  V_{i+1}^\infty(\bsx,\bsu) = \sum_{k=i+1}^\infty g_k(x_k,u_k)
\end{equation*}
and therefore we have that
\begin{equation*}
  W(x_{i+1}) = W(x_i) - g_i(x_i,u_0^\star(x_i)) < W(x_i).
\end{equation*}
This poses the restriction that $g_k$ must be a strictly positive quantity: if a state $\bs{\bar{x}}$ is not weighted in the cost function and $g_k(\bs{\bar{x}})=0$, then $W(x)$ defined above is not strictly decreasing and would not be a Lyapunov function for the asymptotic stability test.

For the finite receding MPC problem with horizon length $K$, we only consider the simpler problem of stability for a stage cost $g$ that is independent from the index $k$. The trick above does not apply because at $i+1$, the new cost to go includes the term $g(x_{i+K})$ and excludes $g(x_i)$
\begin{equation*}
  V^K_{i+1}(\bsu,\bsx) = \sum_{k=i+1}^{i+K}g(x_k,u_k) = V_i^K(\bsu,\bsx) + g(x_{i+K},u_{i+K}) - g(x_i,u_i)
\end{equation*}
and there is no guarantee that the requirement
\begin{equation*}
  W(x_{i+1}) = \min_{\bsu} V^K_{i+1}(\bsu,\bsx^\prime) < \min_{\bsu'} V^K_i(\bsu') = W(x_i)
\end{equation*}
is satisfied.

We can however construct a new optimization problem equal to the original one with the additional constraint $x_K=0$ on last state $x_K$
\begin{equation}
  \label{eq:lyapunov-tracking-MPC}
  \begin{aligned}
    W(x) = \min_{\bsu} &\quad \sum_{k=0}^{K-1} g(x_k,u_k) \\
    \text{subject to } &\quad x_{k+1} = f(x_k,u_k) \\
                                            &\quad  x_0 = x,\ x_K = 0
  \end{aligned}
\end{equation}
based on the intuition that the final state will be close to zero because one is trying to stabilize the system. This is a Lyapunov function if we prove that this is decreasing along a trajectory.

Assume we found the optimal solution $\bsu_i^\star$ to the modified problem for time step $i$ from the initial state $x_{i,0}\neq 0$ that is not already the equilibrium. We construct the suboptimal control input $\bsu_{i+1}$ obtained by taking the optimal solution $\bsu_i^\star$, removing $u_{i,0}^\star$ and adding as the last input $u_{i+1,K-1} = 0$. Using $\bsu_{i+1}$, the trajectory $\bsx_{i+1}$ coincides with $\bsx_i^\star$: in particular, $x_{i+1,K-1}=x_{i,K}^\star=x_K=0$ because of the constraint imposed in eq.~\eqref{eq:lyapunov-tracking-MPC}. The final state\footnote{This does not seem to be relevant for the proof.}
\begin{equation*}
  x_{i+1,K} = Ax_{i+1,K-1}+Bu_{i+1,K-1} = 0.
\end{equation*}

$\bsu_{i+1}$ has a lower associated cost than $\bsu_i^\star$ because we have removed the strictly positive $g_k(x_{i,0}^\star,u_{i,0}^\star)$, since $x_{i,0}\neq 0$; conversely, the suboptimal $\bsu_{i+1}$ has a higher (or equal) cost than the optimal solution $\bsu_{i+1}^\star$:
\begin{align*}
  & W(x_i) \\
  & = V_i^K(\bsx_i^\star, \bsu_i^\star) & \text{definition of $W$} \\
  & = V_{i+1}^K(\bsx_{i+1}, \bsu_{i+1}) + \underbrace{g(x_{i,0}^\star,u_{i,0}^\star)}_{>0} - \underbrace{g(x_{i+1,K-1},u_{i+1,K-1})}_{=0} & \text{definition of $V_{i+1}^K$} \\
  & > V_{i+1}^K(\bsx_{i+1}, \bsu_{i+1}) & \text{drop positive terms} \\
  & \geq V_{i+1}^K(\bsx_{i+1}^\star, \bsu_{i+1}^\star) & \text{optimality of $\bsu_{i+1}^\star$} \\
  & = W(x_{i+1}) & \text{definition of $W$}
\end{align*}
This proves that $W(\cdot)$ is a decreasing function on a trajectory.


%\begin{align*}
%  & W(x_{i+1}) \\
%  & = V_{i+1}^K(\bsx_{i+1}^\star, \bsu_{i+1}^\star) & \text{definition of $W$} \\
%             & \leq V_{i+1}^K(\bsx_{i+1}, \bsu_{i+1}) & \text{optimal trajectory} \\
%             & = \underbrace{V_i^K(\bsx_i^\star, \bsu_i^\star)}_{W(x_i)} - \underbrace{g(x_{i,0},u_{i,0}^\star)}_{>0} + \underbrace{g(x_{i+1,K},u_{i+1,K})}_{=0} & \text{definition of $V_i^K$} \\
%  & < W(x_i)
%\end{align*}

%Assume we found the optimal solution $\bsu_{i+1}^\star$ to the modified problem for $i+1$: this has a lower (or equal) cost than the suboptimal solution $\bsu_{i+1}$ obtained by taking the optimal solution $\bsu_i^\star$ for step $i$, removing input $u_{i,0}^\star$ and adding as the last input $u_{i+1,K-1} = 0$. Using $\bsu_{i+1}$, the trajectory $\bsx_{i+1}$ coincides with $\bsx_i$. The final state
%\begin{equation*}
%  x_{i+1,K} = A\underbrace{x_{i+1,K-1}}_{=x_{i,K}}+Bu_{i+1,K-1} = 0
%\end{equation*}
%because the optimal solution at $i$ has the final state constrained at $x_{i,K}=x_K=0$.

%%%%%%%%%%%%%%%%%%%%%%%%%%%%%%%%%%%%%%%%%%%%%%%%%%%%%%%%%%%%
\section{Implementation of MPC}
\label{sec:implementation-MPC}

An optimal MPC trajectory can be computing using an offline or an online approach.

\subsection{Offline Computation of the Optimal Control Input}
\label{sec:offline-computation-MPC}

The offline method computes the \emph{function} $u_0^\star(\cdot)$. This may require gridding of the solution and is a complex problem because of the exponential number of regions where the solution must be partitioned. To illustrate the approach, in the following section, we will solve a toy problem with time horizon $K=1$ using the KKT condition.

\subsubsection{Short Introduction to the KKT Condition}
\label{sec:short-intro-KKT}

This section is a minimal review of the KKT condition. The minimization problem
\begin{align*}
  \min_x &\ f(x) \\
  \text{subject to } &\ g(x) \le 0
\end{align*}
is equivalent to
\begin{align*}
  \begin{cases}
    \nabla f(x) + \sum_i \mu_i \nabla g_i(x) = 0 \\
    g(x) \le 0 \\
    \mu_i g_i(x) = 0\ \forall i
  \end{cases}
\end{align*}
with $\mu_i\ge 0$ and provided $f$ and $g$ are differentiable.

The condition breaks down into the following two cases:
\begin{itemize}
\item the minimum $\bsx^\star$ is inside the feasible domain as   detemined by $g_i(\bsx^\star) < 0$, in which case $\mu_i=0$ and   $\nabla f(\bsx^\star) = 0$. This is the usual case where the minimum   is found by setting the gradient to zero and the constraints have no   effect, or
\item the minimum occur at the boundary $g_i(\bsx^\star) = 0$,   $\mu_i\neq 0$ and the two vectors $\nabla f(\bsx^\star)$ and   $\sum_i \nabla g_i(\bsx^\star)$ are parallel.
\end{itemize}
The behaviour is qualitatively different whether the minimum is inside or at the boundary.


\subsubsection{Offline computation: a Simple Minimization Problem}

Let us consider the following toy example:
\begin{align*}
  \min_{u_0,x_1} &\ u_0^2+x_1^2 \\
  \text{subject to } &\ x_1 = x_0+u_0 \\
                 &\ x_1\le 1.
\end{align*}
For such simple cases, the KKT condition is all one needs to find minimizers. Eliminating $x_1$ gives the minimization problem
\begin{align*}
  \min_{u_0}\  & f(u_0) \\
  \text{subject to } & g(u_0) \le 0
\end{align*}
where $f(u_0) = u_0^2 + (x_0+u_0)^2$, $g(u_0) = x_0 + u_0 - 1$ and $\nabla_{u_0} f(u_0) + \mu \nabla_{u_0} g(u_0) = 4u_0 + 2x_0 + \mu$. We consider the two cases
\begin{itemize}
\item $\mu = 0$: we have
  \begin{align*}
    & 4u_0 + 2x_0 = 0 \rightarrow u_0 = -\frac{x_0}{2} \\
    & x_0 + u_0 - 1 < 0 \rightarrow x_0 < 2.
  \end{align*}
\item $\mu > 0$: from $\mu g(x) = 0$ the minimum is at the boundary $g(x)=0$ and $4u_0+2x_0+\mu > 4u_0+2x_0$ since $\mu>0$. We have
  \begin{align*}
    &x_0 + u_0 - 1 = 0 \rightarrow u_0 = 1-x_0 \\
    &4u_0 + 2x_0 = 4(1-x_0) + 2x_0 > 0 \rightarrow x_0 > 2.
  \end{align*}
\end{itemize}
Note that the solution is continuous at $x_0=2$: we have therefore the solution
\begin{equation*}
  \begin{cases}
    u_0 = -\frac{x_0}{2} & \text{ for } x_0 < 2 \\
    u_0 = 1-x_0 & \text{ for } x_0 \ge 2.
  \end{cases}
\end{equation*}
In this example, each constraint defines two (affine) regions and for each region there is a different control action. If there were $n$ affine constraints, there would be at most $2^n$ regions, according to the way the constraints overlap. For a large number of constraints, the problem becomes intractables also for offline computation. In practice, one problem to solve is to figure out where the current state is in the state space.

\subsection{Online Computation of the Optimal Control Input}
\label{sec:online-computation-MPC}

Online computation computes the \emph{value} $u_0^\star(\cdot)$ at every iteration and therefore it is typically applied to processes that have constraints and mainly used in slow dynamics. Because the reachable state space is orders of magnitude smaller than the full state space, online planning can significantly reduce storage and computational requirements compared to offline methods~\cite[Sect.~9]{decision-making-kochenderfer}.

\subsection{Other Practical Implementations Aspects}
\label{sec:other-practical-implementation-aspects}

Real world control problems often have an infinite horizon: they are computationally impossible to solve, unless the system has special properties. The finite horizon receding MPC is a computationally tractable approximation to the infinite horizon problem, the approximation being good provided $K$ is chosen large enough that the interesting dynamics of the original problem has a time-scale shorter than $K$.

The ``tail'' of the infinite horizon problem is taken care by the MPC formulation in one of the following ways:
\begin{itemize}
\item by adding a final cost or constraints conditions on the final state: since the point $(x,u)=(0,0)$ is an equilibrium, it is reasonable to impose the constraint $x_K=0$ on the last state;
\item by imposing that $x_K$ live in a small space $\mathcal{X}_K$: if there is a region of the state space sufficiently close to equilibrium that is feasible, the trajectory starting at time $i+1$ will keep the system feasible and hopefully easy to control;
\item by approximating the tail of the infinite horizon problem from element $K+1$ onwards by a final cost $g_K(x_K)$ that is large if the terminal state $x_K$ is far from the equilibrium.
\end{itemize}

We have assumed feasibility (needs to be define or/and give an example of problem that becomes unfeasible) of the MPC problem which may break in the absence of a zero terminal constraint.

%%%%%%%%%%%%%%%%%%%%%%%%%%%%%%%%%%%%%%%%%%%%%%%%%%%%%%%%%%%%
\section{Steady-State Selection}
\label{sec:steady-state-selection}

Until now we have assumed we want to regulate at the steady-state $(x_s,u_s) = (0,0)$. There are situations where we want the plant to operate at a working point $x_s\neq 0$ and to allow for a constant input $u_s$.

If the desired equilibrium $(x_s, u_s)$ is known in advance, no changes to the MPC regulator are necessary except for replacing the stage cost $g_k(x,u)$ by $g_k(x-x_s,u-u_s)$ so that its minimum is at $(x_s,u_s)$.

\subsection{Steady-State Selection from Specifications}
\label{sec:steady-state-from-specs}

Sometimes the problem specifies a working point $(x_\text{spec},u_\text{spec})$ so selected because it is safe and/or economic to operate. This working point may not be a valid state, \textit{i.e} it does not satisfy $x_\text{spec} = f(x_\text{spec},u_\text{spec})$. To find a valid steady-state, one solves offline the optimization problem
\begin{align*}
  \min_{x_s,u_s} & ||x_s-x_\text{spec}||_{Q_s}^2 + ||u_s-u_\text{spec}||_{R_s}^2 \\
  \text{subject to } & x_s = f(x_s,u_s) \\
                 & x_s \in \mathcal{X},\ u_s \in \mathcal{U}.
\end{align*}


\subsection{Incremental Formulation of MPC}
\label{sec:incremental-formulation-MPC}

A common application of feedback is to compensate for unmeasured disturbances to the system such that the selected controlled variables approach their setpoint asymptotically and without offset, even in the case of perfect plant model $x^+ = f(x,u)$. 


a step perturbation cannot be rejected while at the same time reaching the equilibrium point $x=0$ without the presence of an integrator in the plant.


The incremental formulation of MPC uses input variations $\Delta u_k$ instead of the absolute input $u_k$ and augments the state space by adding the control input $
\begin{bmatrix}
  x & u
\end{bmatrix}^\top
$

% Even for LQR there is no steady state tracking error.


In the truck platoon example, keeping the distance to the previous truck constant cannot be relied on only by knowing the (non-linear) relation between gas pedal angle and truck speed, even in presence of feedback. Instead the gas pedal is adjusted to compensate for variations of parameters (engine efficiency, friction, wind gusts\ldots): the controller adjusts $\Delta u_k$ which is integrated to generate the control input $u_k$ that is then fed to the plant. This has also the practical advantage in that it is easier to implement input weights penalizing \emph{changes} in the input signal.

An integrator may already be present in the plant. For example in drug administration, the body integrates the amount of drug (while at the same time slowly using it). (Should it one in this case also measure the integrated $u_k$ since now this is part of the augmented state $x_k,u_k$?)

\subsection{Disturbance Rejection in MPC}
\label{sec:disturbance-rejection}

% Aside from the MPC controller and the plant, a MPC-based control system may include a state estimator and a steady-state target selection. The estimator estimates the true state based on $(x_k,u_k)$ that are corrupted by the disturbance $d_k$ if the there is a model of the disturbance.


%%% Local Variables:
%%% mode: latex
%%% TeX-master: "notes"
%%% End:
